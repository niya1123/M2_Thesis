\section{本章の概要}
本章では,本研究に関連する研究について述べる.
\ref{sec:concept_ref}節ではコンセプトマップを用いた研究について述べる.
\ref{sec:kasika_ref}節では学習成果の可視化に関する研究について述べる.
\ref{sec:gra_ref}節ではグラフデータベースを用いた研究について述べる.
\ref{sec:my_thesis}節では,他研究と本研究にどのような違いがあるかを述べる.

\section{コンセプトマップを用いた研究}\label{sec:concept_ref}
コンセプトマップを用いた研究には,東本氏らの研究\cite{toumoto}と野村氏らの研究\cite{nomura_manabu}が挙げられる.
東本氏らの研究では,コンセプトアップを作成した学習者に対しフィードバックを返すことは重要であるが,決して容易ではないとしている.
第一に個別診断の困難性,第二に仮に診断をしても誤りをフィードバックしたとき,学習者の解答を否定し正解を提示する否定的フィードバックでは学習効果が低く,学習者はコンセプトマップの誤りを受け入れないか,なぜ誤りなのかを考えずに修正する可能性があり,自発的な誤り修正ができない点が挙げられる.
そこで,東本氏らは階層構造の理解の促進を目的とした学習者自身によるコンセプトマップの構築のためのシステム開発を行った.
特に,構築したコンセプトマップに対し,個別診断を行い,誤りがあれば誤りの可視化によるフィードバックを与えることとした.
東本氏らが提案した可視化手法は,各属性の意味を学習支援システムに組み込むため,汎用性が乏しいものとなった.
しかし,可視化を段階的に行うことにより,様々な階層性に対してもXMLや画像を用いれば可視化できるとしている.

野村氏の研究では,コンセプトマップはこれまで多くの教師が学校教育に取り入れており,コンセプトマップはある特定の学習過程において,教師と学習者が焦点化する必要のある少数のアイデアを明確にし,概念的意味を結びつける視覚的地図により,学習課題の達成後の図式的な要約を提供するものとしている.
しかし,コンセプトマップは学習中においても有効であるが最も友好的に利用できる可能なのは学習後の学習者の知識構成の表出であると主張している.
これはNovak\cite{concept}\cite{novak}が有意味学習の評価ツールとしてコンセプトマップが有効であるとしている点でも同様のことがいえるとしている.
そこで野村氏らはコンセプトマップを利用した学習ではなく,一定のまとまりのある学習内容を学習した結果をコンセプトマップを利用して評価することを想定とした,コンセプトマップを利用した学習評価支援システムを開発した.
このシステムでは,コンセプトマップの作成作業時間を短縮,管理を支援できる.

\section{学習成果の可視化に関する研究}\label{sec:kasika_ref}
学習成果の可視化に関する研究として,平塚氏らの研究が挙げられる.