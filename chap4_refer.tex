\section{本章の概要}
本章では,本研究に関連する研究について述べる.
\ref{sec:concept_ref}節ではコンセプトマップを用いた研究について述べる.
\ref{sec:kasika_ref}節では学習成果の可視化に関する研究について述べる.
\ref{sec:my_thesis}節では,他研究と本研究にどのような違いがあるかを述べる.

\section{コンセプトマップを用いた研究}\label{sec:concept_ref}
コンセプトマップを用いた研究には,東本氏らの研究\cite{toumoto}と野村氏らの研究\cite{nomura_manabu}が挙げられる.
東本氏らの研究では,コンセプトアップを作成した学習者に対しフィードバックを返すことは重要であるが,決して容易ではないとしている.
第一に個別診断の困難性,第二に仮に診断をしても誤りをフィードバックしたとき,学習者の解答を否定し正解を提示する否定的フィードバックでは学習効果が低く,学習者はコンセプトマップの誤りを受け入れないか,なぜ誤りなのかを考えずに修正する可能性があり,自発的な誤り修正ができない点が挙げられる.
そこで,東本氏らは階層構造の理解の促進を目的とした学習者自身によるコンセプトマップの構築のためのシステム開発を行った.
特に,構築したコンセプトマップに対し,個別診断を行い,誤りがあれば誤りの可視化によるフィードバックを与えることとした.
東本氏らが提案した可視化手法は,各属性の意味を学習支援システムに組み込むため,汎用性が乏しいものとなった.
しかし,可視化を段階的に行うことにより,様々な階層性に対してもXMLや画像を用いれば可視化できるとしている.

野村氏の研究では,コンセプトマップはこれまで多くの教師が学校教育に取り入れており,コンセプトマップはある特定の学習過程において,教師と学習者が焦点化する必要のある少数のアイデアを明確にし,概念的意味を結びつける視覚的地図により,学習課題の達成後の図式的な要約を提供するものとしている.
しかし,コンセプトマップは学習中においても有効であるが最も友好的に利用できる可能なのは学習後の学習者の知識構成の表出であると主張している.
これはNovak\cite{concept}\cite{novak}が有意味学習の評価ツールとしてコンセプトマップが有効であるとしている点でも同様のことがいえるとしている.
そこで野村氏らはコンセプトマップを利用した学習ではなく,一定のまとまりのある学習内容を学習した結果をコンセプトマップを利用して評価することを想定とした,コンセプトマップを利用した学習評価支援システムを開発した.
このシステムでは,コンセプトマップの作成作業時間を短縮,管理を支援できる.

\section{学習成果の可視化に関する研究}\label{sec:kasika_ref}
学習成果の可視化に関する研究として,平塚氏ら\cite{hira}と手塚氏ら\cite{teduka}の研究が挙げられる.
平塚氏らの研究では,高等教育機関において学生自身が授業の学習成果を把握するものは,成績評定,GPA,資格取得などがあるとしているが,これらは授業単位を習得したという結果のみを計るものであり,学生にとって自分にどのような力が身についたのかわかりづらいとしている.
また,このことから学生は学習においての将来設計も難しくなり,授業の取り組みも消極的になるなど悪循環に陥ってしまう.
そこで,平塚氏らは学習成果の到達度を可視化し,学生に分かりやすい形で提示し自己効力感を得て学習意欲向上のために,教務システムとeポートフォリオを連携した学習成果可視化システムを構築した.
すでに同様のシステムを独自開発している事例はあるが,オープンソースのシステムを用いて構築し,公開・フィードバックする点が研究の意義としている.

手塚氏らの研究では,高等教育における振り返りは他者と関わりあいながら自主的に学び続けるために必要な能力として注目されているとし,構成主義的な学習観において教員が何を教えるかから学習者が何を学び取るかへと視点の転換が主張されていることから,学習中に期待通り成果が得られたのかどうかを常に振り返り,成功または失敗の要因を学習者が認識することが重要であるとしている.
しかし,全学習者が適切に振り返りを行えるとは限らないため,平塚氏らは振り返りの質的向上を目的とし,期末試験の予測得点と学習者データの可視化による振り返り支援システムを提案・開発した.
振り返りシステムでは,基礎数学の振り返りシートを分析し,eラーニングのヒント閲覧回数,eラーニングの学習時間,Vマーク式学習法におけるVの数が理解度の向上に結びつく振り返りに役立つことが考えられため,それらを折れ線グラフで時系列順に可視化するシステムを作成した.
これにより学習者は可視化機能を用いて学習でき,さらに各学習者がどのような学習データを見ながら振り返りを行っているのかというログが収集できる.

\section{本研究の特徴}\label{sec:my_thesis}
いままで上げてきた研究はどれも実際の授業の中でコンセプトマップや学習者の学習進捗を可視化しているものが多い.
一方,本研究では,eラーニングで学習する学習者に重きを置いている.
eラーニングは確かに授業の一環として用いられることもあるが,基本的には学習者一人で課題をこなしていくものが多い.
また,eラーニングにおけるフィードバックは否定的フィードバックが多く,何故間違えたのかという思考に至る可能性が低くなる.
加えて,今までの研究成果物はそのシステム上でのみ動作する物が多い.
本研究ではeラーニング上で学習する学習者に対して指導者が直接かかわらずとも学習者自身だけで,自身の学習理解度を把握し,学習目標を設定できる.
また,本システムはグラフデータの作成,削除や可視化に至るデータ取得に関して全てAPI化して実装している.
このことにより,本システムの可視化表現方法は本システムのみで使用できる機能となっているが,コンセプトマップの情報をグラフデータに変換し保存,削除,またデータ呼び出しに関してはAPIを通じて実行できる.
このことから,可視化に至るまでのデータ取得は任意のアプリケーションでも実施できるため,様々なeラーニングシステムで本システムの機能を実行できる.


