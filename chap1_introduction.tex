\section{本章の概要}
本章では,研究背景と研究目的,評価実験の概要,そして本論文の構成について記載する.

研究背景では,教育の情報化に関する課題,本研究に関する研究について記載している.

研究の目的では,本研究の目的を記載している.

研究の内容では,開発したシステムとシステム各部の機能の説明を記載している.

評価実験の概要では,システムを評価する為実施した評価実験の内容と結果を記載している.

また,本論文の構成では各章を番号付きでリストで記載している.

\section{研究背景}
2019 年 12 月に文部科学省が作成した「教育の情報化に関する手引き」\cite{tebiki}によると教育の情報化が促進されている.
e ラーニング\cite{e}は「情報通信技術の時間的・空間的制約をなくす」,「双方向性を有する」,「カスタマイズを容易にする」という特性を有するシステムのうちの一つであることから,教育の情報化に有効である.
e ラーニング上で学習するにあたり,自身の学習目標を設定することは,学びを深める手段のうちの一つである\cite{seman}.

一方,学習目標を設定するには,自身が学習したい対象の知識を把握している必要がある.
しかし,学習者自身では学習項目を理解していると主観的には考えていても,他人が客観的に判断すると理解できていない場合があり,学習者自身で学習目標を設定することは必ずしも容易ではない.

東本氏らの研究では,科学領域においては習得すべきさまざまな概念および概念間の関係が存在し,その一つに概念の階層構造を学習者に理解させることは科学の学習において重要な課題であると認識していた.
そこで,階層構造の理解の促進を目的とした学習者自身によるコンセプトマップ(以下,CMap)\cite{concept}の構築のためのシステムを開発した\cite{toumoto}.

野村氏らの研究では,学習方法の一つとして,学習した内容を整理して他の学習者に教える事で自信の理解を深める方法があり,
他者に対して学習内容を理解させることができるか否かで自身の理解が十分であるか否かを学習者自身が再確認することができるという教え合い学習をシステムの推奨を用いて実際に学習者間で行わさせることを目的とした研究を進めている\cite{nomura}.

平塚氏らの研究では,高等教育機関における学生たちに対して,教育課程を理解してもらうことが重要であると考え,教務システムとeポートフォリオを連携した「学習成果可視化システム」を構築した.このシステムはオープンソースのシステムを用いて構築し,公開・フィードバックする点で意義があるとしている\cite{hira}.

西川氏らの研究では,大学における学生がディプロマポリシーに向けて現段階でどのような学修を積み立てているのか確認することを目的とした,
グラフデータベースNeo4jによる学習ポートフォリオ作成支援システムを開発している\cite{nisi}.この研究で,ディプロマポリシーに向けて学修をどのように積み立てているかを可視化でき,
それによりディプロマポリシーに向けた学修達成度を把握でき,学生がその後どのように履修計画を立案するかの指標となることが示された.

\section{研究の目的}
本研究では,CMapを用いて学習目標を設定する学習者を対象としその学習目標設定の支援を目的としている.

\section{研究の内容}
本研究では,グラフデータベースを用いて学習者の理解度を可視化し,学習目標の設定を支援できるグラフデータベースを用いた学習者理解度可視化システム(以下,本システム)を用いた学習者理解度可視化システムを開発した.

学習目標を設定するには自身の学習度合いを正確に把握必要がある.しかし,自身の学習度合いを主観的に把握できても客観的に見ると誤っている可能性がある.
そこで本システムのグラフデータ可視化機能により学習者のテストの回答情報と,指導者による学習目標,学習項目の情報からグラフデータベースを用いてCMapを自動的に
作成することにより,学習者自身が作成したCMapと本システムが自動的に作成したCMapを比較することにより,客観的に学習者の理解度を把握できる.

グラフデータでCMapを作成するにあたり,本システムにはグラフデータ管理機能とグラフデータ入力補助機能が存在する.
グラフデータ管理機能はグラフデータを管理する機能で,WebAPIを用いてグラフデータを管理できるため,様々なアプリケーションでAPIを用いることにより,グラフデータを管理できる.
グラフデータ入力補助機能では,本システムを用いて学習者指導する指導者に対して,学習目標・学習項目の入力を容易に実施するための機能である.
グラフデータ入力補助機能はフォームが木構造で入力することが可能で,学習目標・学習項目の入力が容易にできる.また,グラフデータ入力補助機能はグラフデータ管理機能のAPIを用いることにより,学習目標,学習項目をグラフデータへと変換し,グラフデータベースへとグラフデータを保存,および呼び出しを実行している.


\section{評価実験の概要}
グラフデータ可視化機能を使って,座学における学習目標設定方法と本システムを用いた学習目標設定方法を比較し,有用性を検証した.
検証には,Googleフォームを用い,本システム利用者群と本システム非利用者群にグループ分けを行い,事前テストと事後テスト,アンケートを用いた利用評価実験を実施し,
座学における学習目標設定方法と本システムを用いた学習目標設定方法のどちらがより良い結果になったかを確認した.

\section{本論文の構成}
本論文の以降の章では,本研究の具体的な内容について述べる.
第章では,コンセプトマップについて述べる.
第章では,キットビルド概念マップについて述べる.
第章では,本研究に関連している研究について述べる.
第章では,本システムの要件について述べる.
第章では,学習者理解度可視化システムについて述べる.
第章では,評価実験について述べる.
第章では,本研究の結論について述べる.
