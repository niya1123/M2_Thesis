\section{序論}\label{sec1}
\subsection{本章の概要}
本章では,研究背景と研究目的,評価実験の概要,そして本論文の構成について記載する.
研究背景では,教育の情報化に関する課題,本研究に関する研究について記載している.
研究の目的では,本研究の目的と本研究の内容を記載している.
本研究の内容では,開発したシステムとシステム各部の機能の説明を記載している.
また,本論文の構成では各章を番号付きでリストで記載している.

\subsection{研究背景}
2019 年 12 月に文部科学省が作成した「教育の情報化に関する手引き」\cite{tebiki}によると教育の情報化が促進されている.
e ラーニング\cite{e}は「情報通信技術の時間的・空間的制約をなくす」,「双方向性を有する」,「カスタマイズを容易にする」という特性を有するシステムのうちの一つであることから,教育の情報化に有効である.
e ラーニング上で学習するにあたり,自身の学習目標を設定することは,学びを深める手段のうちの一つである\cite{seman}.
一方,学習目標を設定するには,自身が学習したい対象の知識を把握している必要がある.
しかし,学習者自身では学習項目を理解していると主観的には考えていても,他人が客観的に判断すると理解できていない場合があり,学習者自身で学習目標を設定することは必ずしも容易ではない.

東本氏らの研究では,科学領域においては習得すべきさまざまな概念および概念間の関係が存在し,その一つに概念の階層構造を学習者に理解させることは科学の学習において重要な課題であると認識していた.
そこで,階層構造の理解の促進を目的とした学習者自身によるコンセプトマップ(以下,CMap)\cite{concept}の構築のためのシステムを開発した\cite{toumoto}.

野村氏らの研究では,学習方法の一つとして,学習した内容を整理して他の学習者に教える事で自信の理解を深める方法があり,
他者に対して学習内容を理解させることができるか否かで自身の理解が十分であるか否かを学習者自身が再確認することができるという教え合い学習をシステムの推奨を用いて実際に学習者間で行わさせることを目的とした研究を進めている\cite{nomura}.

平塚氏らの研究では,高等教育機関における学生たちに対して,教育課程を理解してもらうことが重要であると考え,教務システムとeポートフォリオを連携した「学習成果可視化システム」を構築した.このシステムはオープンソースのシステムを用いて構築し,公開・フィードバックする点で意義があるとしている\cite{hira}.

西川氏らの研究では,大学における学生がディプロマポリシーに向けて現段階でどのような学修を積み立てているのか確認することを目的とした,
グラフデータベースNeo4jによる学習ポートフォリオ作成支援システムを開発している\cite{nisi}.この研究で,ディプロマポリシーに向けて学修をどのように積み立てているかを可視化でき,
それによりディプロマポリシーに向けた学修達成度を把握でき,学生がその後どのように履修計画を立案するかの指標となることが示された.
\subsection{研究の目的}

\subsection{評価実験の概要}

\subsection{本論文の構成}
