本システムは,コンテナ仮想化プラットフォームであるDockerを基盤として開発している.
指導者は,本システムの運用Webサーバが内包されたDockerイメージを自身のPC上にインポートして学習用コンテナを生成することで,Webアプリケーションを実行でき,学習者に対してコンセプトマップを用いた学習を実施できる.
また,本システムにおけるグラフデータベースであるNeo4jとPythonのWebフレームワークの一つであるFlaskを用いることによりグラフデータベースとコンセプトマップのデータ変換部をAPIを用いて作成していることにより,様々なプラットフォームで本システムのグラフデータべース+コンセプトマップという学習環境を利用する事が可能である.
以下に,本システムの利用にあたって指導者が実行しなければならないDockerコマンドを一覧に示す.

\begin{itemize}
    \item docker-compose up -d --build\\
    - 指導者のPC上に本システムのDockerイメージをインポートし実行する
    
    \item docker-compose down -v\\
    - 本システムに何か変更を加えた際にコンテナを停止させるためのコマンド

    \item docker system prune\\
    - 本システムに何か変更を加えた際にコンテナに残っているキャッシュを削除するためのコマンド
    
\end{itemize}
