本システムの運用における負担として,グラフデータの入力作業と学習者の試験結果の得点入力作業の2つの負担要素が考えられる.
まず,グラフデータの入力作業における負担について述べる.

グラフデータの入力作業において,現在のシステムでは複数の学習分野に関するグラフデータの入力が不可能である.
しかし,一つの学習分野に絞ってグラフデータを作成する場合は本システムはグラフデータ入力補助機能により容易にグラフデータを入力できる.
このことから複数の学習分野を教える必要がある指導者にとっては本システムはやや大きい負担がかかることが考えられるが,一分野に関して指導する場合においては少ない負担になるのではと考えられる.

一方,学習者の試験結果の得点入力作業は非常に大きいと考えられる.この理由として現在のシステムでは複数の学習分野,すなわちノードに対して一括で得点を入力する機能がないことが挙げられる.
これにより指導者は学生の数x学習分野の数だけ得点データを入力しなければならず非常に負担がかかると考えられる.

上記の理由により,指導者のグラフデータ入力作業の負担はやや大きいが,学習者の試験結果の得点入力作業に関しては非常に大きな負担であると考えられる.
このため本システムでは,機能としてその負担を減らす機能は作成が不十分だったが,入力を補助するスクリプトを作成することにより今回の負担を軽減した.
今後本システムを開発するにあたってはこの負担を減らすような機能を作成できればと考えている.