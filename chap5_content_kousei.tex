本システムの構成を図\ref{fig:kousei}に示す.
本システムは,学習者がインターネットを通じて本システムのWebアプリケーションに接続することによってコンセプトマップを作成し学習を進めることができる.
本システムはすべての開発環境をDockerを用いて作成しているため,Dockerを使用できる環境であればどこでも本システムを実行することが可能である.

本システムで利用可能な学習教材は,その学習分野において教材提供者が階層構造を持つと判断した学習教材であれば利用可能である.
コンセプトマップを作成するにあたって,本システムではグラフデータベースを用いてコンセプトマップにおけるノードやリンク,リンクキーワードをグラフデータベースにおけるノード,エッジ,プロパティに変換することによってグラフデータベースにコンセプトマップのデータを保存し,その後グラフデータベースを可視化するライブラリを用いてコンセプトマップを閲覧,作成できる機能を作成した.
本システムを利用した学習手順として,学習者は指導者が作成した学習コンテンツを学び,そのフィードバックとして各問題がどのような学習分野となるのか教授してもらう.
その後,本システムを用いてそれぞれの学習分野をコンセプトマップを用いてどのような階層構造になっているのかを予測しながらコンセプトマップを作成する.
最後に本システムが指導者から入力された学習分野に対する階層構造のデータからコンセプトマップを自動的に作成する.
これにより学習者は,自身が作成したコンセプトマップと,自動的に作成されたコンセプトマップとを比較することにより,学習分野における自身の階層構造に対する理解を深めることができる.
加えて本システムではコンセプトマップのノードにおいて学習者の問題の回答情報から点数によってノードの背景色を表示できる.
これにより学習者は対象学習分野について,主観的に考えていた学習理解度と実際のテストの点数による学習理解度をはっきりと視覚的に確認でき,自分は特定分野においてしっかり学修できていたと思っていたが,実際はあまり理解できてきなかったという勘違いを正すことができる.
\begin{figure}[htbp]
\begin{center}
\includegraphics[width=10cm]{img/kousei.eps}
\end{center}
\caption{システム構成}
\label{fig:kousei}
\end{figure}