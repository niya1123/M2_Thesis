e ラーニング上で学習するにあたり,自身の学習目標を設定することは,学びを深める手段のうちの一つである.
一方,学習目標を設定するには,自身が学習したい対象の知識を把握している必要がある.
しかし,学習者自身では学習項目を理解していると主観的には考えていても,他人が客観的に判断すると理解できていない場合があり,学習者自身で学習目標を設定することは必ずしも容易ではない.

そこで本研究では,学習目標の設定支援を目的に,学習者の理解度を可視化する,グラフデータベース を用いた学習者理解度可視化システムを開発した.
本システムは e ラーニングで学習している学習者を対象としたシステムで,コンセプトマップ(以下,CMap)を利用して学習者が学習目標を設定する場合に本システムを利用することを想定している.
学習者は指導者が作成した問題を解き,本システムを用いてCMapを作成する.
本システムでは学習者の回答情報からCMapを作成し,学習者は自身が作成したCMapと,システムが生成したCMapを比較することにより,自身の学習理解度を客観的に確認でき,学習目標設定の基準にできる.

開発した本システムを評価するために,学習者が座学における学習目標設定方法による学習と本システムのグラフデータ可視化機能を用いた学習目標設定方法による学習を比較し,有用性を検証するための事前事後テストによる評価実験とアンケートによる利用評価実験を実施した.
座学と比較した本システムによる学習の有用性検証実験では,情報学科の学生と情報学科を卒業した大学院生16名に対して基本情報技術者試験の午前試験を基にした,事前テスト事後テストよる座学と比較した本システムによる学習の有用性を検証した.
結果として,座学すなわちシステム非利用者群の事前テストと事後テストの平均点が0.75 点増えているのに対し,本システム利用者群の平均点は2.125 点上昇していた.
このことから,座学で学習したシステム非利用者群と比較して,本システムを利用して学習したシステム利用者群の方がより高い点数が得られたことが確認できた.

また,本システムの利用評価実験では,システム利用者群はシステム非利用者群と比べて,自分の主観的学習理解度と実際に返却された学習理解度に差がなかったことが表されている.
すなわち,システム利用者の方がシステム非利用者と比べて,学習者自身の学習内容に対する主観的理解度と客観的理解度に差がなかったことが分かった.
この結果から,システム利用者群は正しく自身の学習進行度を理解できていたため,システム非利用者群より事後テストの得点が上昇していたのではと考えられる.

本研究では,e ラーニングにおける学習目標設定支援を目的に,学習者の理解度をCMapで可視化する,グラフデータベースを用いた学習者理解度システムを開発した.
実験の結果座学で学習したグループと比較して,本システムを利用して学習したグループの方がより有意に点数が高いことが確認できた.
このことから本システムは学習目標設定支援が可能である.