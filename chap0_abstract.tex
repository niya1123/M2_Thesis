\begin{titlepage}
\begin{center}
\vspace*{1cm}
\large
{\Huge 修\hspace{2zw}士\hspace{2zw}論\hspace{2zw}文}\\
\vspace*{1cm}
{\huge 令和五年度}\\
\vspace*{2cm}
{\huge 論文内容の要旨}\\
\vspace*{1cm}
{\LARGE グラフデータベースを用いた\\
学習者理解度可視化システムの開発
}\\
\vspace*{7cm}
\LARGE{近畿大学大学院\\
総合理工学研究科\\
エレクトロニクス系工学専攻\\
21-3-334-0415番 \hspace{0.5zw} 栗\hspace{0.5zw}岡\hspace{0.5zw}陽\hspace{0.5zw}平
}
\end{center}
\end{titlepage}

\newpage

\begin{titlepage}

    %背景
    2019 年 12 月に文部科学省が作成した「教育の情報化に関する手引き」\cite{tebiki}によると教育の情報化が促進されている.
    e ラーニングは「情報通信技術の時間的・空間的制約をなくす」,「双方向性を有する」,「カスタマイズを容易にする」という特性を有するシステムのうちの一つであることから,教育の情報化に有効である.
    e ラーニング上で学習するにあたり,自身の学習目標を設定することは,学びを深める手段のうちの一つである[1].
    一方,学習目標を設定するには,自身が学習したい対象の知識を把握している必要がある.
    しかし,学習者自身では学習項目を理解していると主観的には考えていても,他人が客観的に判断すると理解できていない場合があり,学習者自身で学習目標を設定することは必ずしも容易ではない.

    %関連研究

    %本研究
    そこで本研究では,学習目標の設定支援を目的に,学習者の理解度を可視化する,グラフデータベース を用いた学習者理解度可視化システムを開発した.
    本システムは e ラーニングで学習している学習者を対象としたシステムで,コンセプトマップ(以下,CMap)を利用して学習者が学習目標を設定する場合に本システムを利用することを想定している.
    学習者は指導者が作成した問題を解き,本システムを用いて CMap を作成する.
    本システムでは学習者の回答情報から CMapを作成し,学習者は自身が作成した CMap と,システムが生成した CMap を比較することにより,自身の学習理解度を客観的に確認でき,学習目標設定の基準にできる.
    
    %実験
    
\end{titlepage}