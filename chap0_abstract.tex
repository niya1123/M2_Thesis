\begin{titlepage}
\begin{center}
\vspace*{1cm}
\Large
{\Huge 修\hspace{2zw}士\hspace{2zw}論\hspace{2zw}文}\\
\vspace*{1cm}
{\huge 令和五年度}\\
\vspace*{2cm}
{\huge 論文内容の要旨}\\
\vspace*{1cm}
{\LARGE グラフデータベースを用いた\\
学習者理解度可視化システムの開発
}\\
\vspace*{7cm}
\LARGE{近畿大学大学院\\
総合理工学研究科\\
エレクトロニクス系工学専攻\\
21-3-334-0415番 \hspace{0.5zw} 栗\hspace{0.5zw}岡\hspace{0.5zw}陽\hspace{0.5zw}平
}
\end{center}
\end{titlepage}

\newpage

\begin{titlepage}

    %背景
    2019 年 12 月に文部科学省が作成した「教育の情報化に関する手引き」\cite{tebiki}によると教育の情報化が促進されている.
    e ラーニング\cite{e}は「情報通信技術の時間的・空間的制約をなくす」,「双方向性を有する」,「カスタマイズを容易にする」という特性を有するシステムのうちの一つであることから,教育の情報化に有効である.
    e ラーニング上で学習するにあたり,自身の学習目標を設定することは,学びを深める手段のうちの一つである\cite{seman}.
    一方,学習目標を設定するには,自身が学習したい対象の知識を把握している必要がある.
    しかし,学習者自身では学習項目を理解していると主観的には考えていても,他人が客観的に判断すると理解できていない場合があり,学習者自身で学習目標を設定することは必ずしも容易ではない.

    %関連研究
    東本氏らの研究では,科学領域においては習得すべきさまざまな概念および概念間の関係が存在し,その一つに概念の階層構造を学習者に理解させることは科学の学習において重要な課題であると認識していた.
    そこで,階層構造の理解の促進を目的とした学習者自身によるコンセプトマップ(以下,CMap)\cite{concept}の構築のためのシステムを開発した\cite{toumoto}.
    野村氏らの研究では,学習方法の一つとして,学習した内容を整理して他の学習者に教える事で自信の理解を深める方法があり,
    他者に対して学習内容を理解させることができるか否かで自身の理解が十分であるか否かを学習者自身が再確認することができるという教え合い学習をシステムの推奨を用いて実際に学習者間で行わさせることを目的とした研究を進めている.\cite{nomura}
    平塚氏らの研究では,高等教育機関における学生たちに対して,教育課程を理解してもらうことが重要であると考え,教務システムとeポートフォリオを連携した「学習成果可視化システム」を構築した.このシステムはオープンソースのシステムを用いて構築し,公開・フィードバックする点で意義があるとしている.

    %本研究
    本研究では,学習目標の設定支援を目的に,学習者の理解度を可視化する,グラフデータベース を用いた学習者理解度可視化システムを開発した.
    本システムは e ラーニングで学習している学習者を対象としたシステムで,CMapを利用して学習者が学習目標を設定する場合に本システムを利用することを想定している.
    学習者は指導者が作成した問題を解き,本システムを用いて CMap を作成する.
    本システムでは学習者の回答情報から CMapを作成し,学習者は自身が作成した CMap と,システムが生成した CMap を比較することにより,自身の学習理解度を客観的に確認でき,学習目標設定の基準にできる.
    
    CMapを作成するにおいて,事前に学習目標と学習項目の情報が必要になる.そこで本システムでは,CMapをキットビルド概念マップシステム\cite{kit}を用いて作成しており,
    事前に学習者を指導する指導者が本システムを用いてCMapに関する学習目標,学習項目を入力できる.これにより,学習者は事前に設定された学習目標,学習項目を基に本システムを用いて自身でCMapを作成できる.

    本システムには,グラフデータ管理機能,グラフデータ入力補助機能,グラフデータ可視化機能が存在し,
    グラフデータ管理機能では,グラフデータベースの内の一つであるNeo4jを用いてグラフデータを管理し,学習者・指導者にグラフデータを通してCMapの学習目標,学習項目を提供する.
    グラフデータ入力補助機能は指導者がCMapの学習目標,学習項目を入力するときに,そのデータを木構造で入力できるフォームを用意し,そのデータをグラフデータへと変換しNeo4jに登録する機能である.
    グラフデータ可視化機能緒は,学習者に対して指導者が入力したCMapのグラフデータを可視化し提供する機能である.また,学習者がCMapを作成するためのフォームも提供している.

    %実験
    本研究では,グラフデータ可視化機能を用いて学習者に座学における学習目標設定方法と本システムを用いた学習目標設定方法を比較するための確認テストによる評価実験を実施した.
    実験の結果,本システム使用者と本システム非利用者の学習目標設定方法では本システム利用者の学習目標設定方法の方が確認テストの結果が向上していることを確認した.
\end{titlepage}