\section{本章の概要}
本章では,開発したシステムの評価実験とその考察について述べる.
そして本研究では,学習者が座学における学習目標設定方法による学習と本システムのグラフデータ可視化機能を用いた学習目標設定方法による学習を比較し,
有用性を検証するための事前事後テストによる評価実験とアンケートによる利用評価実験を実施した.

\section{座学と比較した本システムによる学習の有用性検証実験}
実験では,基本情報技術者試験の午前試験を基にした,事前テスト事後テストによる座学と比較した本システムによる
学習の有用性を検証した.

\subsection{実験対象者}
実験対象者は,情報学科の学生と情報学科を卒業した大学院生(4年生:4名,修士1年生:7名,修士2年生:5名)の16名を実験協力者とした.
当学科では,3年生までにプログラミングは勿論ながら様々な情報学に関しての講義があるため,基本情報処理技術者試験の午前試験に関する知識も備えている.

\subsection{実験準備}
本実験を実施するにあたり,基本情報技術者試験の午前試験に関する事前テスト・事後テストを用意した.
学習教材としては事前テストに解説を付属させておりそれを学習教材とした.

事前テスト,事後テストはともに10点満点として,事前テスト・事後テストは同レベルの別の問題を使用した.
事前テストと事後テストはGoole From上で解答してもらった.
事前テストの内容を図\ref{fig:jizen1}~図\ref{fig:jizen5}に示す.
事後テストの内容を図\ref{fig:jigo1}~図\ref{fig:jigo5}に示す.
学習教材の例を図\ref{fig:kyozai1},図\ref{fig:kyozai2}に示す.
