\section{本章の概要}
本章では,本研究で使用している技術について述べる.

\ref{sec:docker}節では,本システムの構築とPC上で利用するDockerについて述べる.
\ref{sec:python}節では,本システムで使用するサーバサイド言語のPythonとそのライブラリであるFlaskとDjangoについて述べる.
\ref{sec:neo4j}節では,本システムで使用するグラフデータベースであるNeo4jについて述べる.
\ref{sec:cyto}節では,本システムで利用するグラフデータベース可視化ライブラリであるCytoScapeについて述べる.

\section{Docker}\label{sec:docker}
Docker\cite{docker}は,Docker社が開発してるコンテナ仮想化を用いてアプリケーションの開発や実行をするためのオープンプラットフォームである.コンテナ仮想化とは,ホストOSのカーネルを利用してプロセスやユーザを隔離して動かすことで開発環境や実行環境を構築するためのものである.ハイバーバイザを利用した仮想化ソフトウェアとは違い,ハードウェアを仮想化しないため軽量で高速に起動や停止することが可能である.

本システムでは,Dockerを利用してWebアプリケーションを構築している.このため,実運用されているPC環境に影響を及ぼさずに本システムが利用可能となっている.またDockerイメージの共有をすることで特定のOS依存することなく本システムの可能となっている.
また,本システムではDockerを利用するためにComposeというものを用いている.
Composeとは,複数のコンテナを定義し実行するDockerアプリケーションの為のツールである.
ComposeにおいてはYAML\cite{YAML}ファイルを使ってアプリケーションサービスの設定を行う.
コマンド一つ実行するだけで,設定内容に基づいたアプリケーションの生成,起動を行う.

Composeを使うには基本的に3つのステップを踏む.

\begin{enumerate}
    \item アプリケーション環境をDockerfileに定義する.
    \item アプリケーションを構成するサービスをdocker-compose.ymlファイル内に定義する.
    \item docker-compose upを実行したら,Composeはアプリケーション全体を起動・実行する.
\end{enumerate}

今回,本システムで利用するDockerコマンドについて表\ref{tab:docker_cmd}に示す.

\begin{table}[htb]
\centering
\caption{Dockerコマンド一覧}
\label{tab:docker_cmd}
\begin{tabular}{|c|c|}  \hline
    docker-compose up --build & コンテナのビルドと起動 \\ \hline
    docker-compose down -v & 起動したコンテナの削除 \\ \hline
    docker-compse exec [サービス名] [コマンド] & 起動したコンテナにログインしてコマンドを発行 \\ \hline		  
\end{tabular}
\end{table}

\section{Python}\label{sec:python}
Python\cite{python}はWindows,Linux/Unix,Mac OS X などの主要なオペレーティングシステムおよびJavaや.NETなどの仮想環境でも動作するインタプリタ形式の,対話的な,オブジェクト指向プログラミング言語である.
この言語には,モジュール,例外,動的な型付け,超高水準の動的データ型,およびクラスが取り入れられている.
Pythonはオブジェクト指向プログラミングを超えて,手続き型プログラミングや関数型プログラミングなど複数のプログラミングパラダイムをサポートしている.

\subsection{Django}
Django\cite{Django}は,Pythonで実装された無料オープンソースのWebアプリケーションフレームワークの一つである.
Djangoが作られた時の目的として,複雑なデータベース主体のウェブサイトを簡単に構築するというものがある.
これを実現するため,Djangoではコンポーネントの再利用性,素早い開発の原則に力を入れている.
本システムでは,webサーバとurlルーティングで利用している.

\subsection{Flask}
Flask\cite{flask}は,Pythonで実装された軽量のWSGI\cite{wsgi}ウェブアプリケーションフレームワークである.
Flaskはルーティング・リクエスト処理・blueprintモジュールの機能がある.
Flask自体はこの3つの機能しかなく,拡張機能を個別に追加することで様々な機能を利用できる.
本システムでは,API作成において利用している.

\section{Neo4j}\label{sec:neo4j}
Neo4j\cite{neo4j}は,グラフ構造のデータモデルを扱うデータベース管理システムであり,グラフデータベースに分類される.
Neo4jは,グラフデータベースの中で最も利用されている製品の一つでありアメリカのNeo Technology社によって開発されている.
本システムでは,コンセプトマップをWebアプリケーションで利用するために,コンセプトマップをグラフデータへと変換し保存している.
コンセプトマップは階層構造を持つ概念の理解に有効であるが,コンセプトマップをシステム上で扱う場合,関係性を検索しつつ対象知識を取り出そうとすると少なくない検索時間がかかる.
そこでグラフデータベースであるNeo4jを用いることにより,この検索時間を短縮できるため今回本システムで利用することとした.

\section{CytoScape}\label{sec:cyto}
CytoScape\cite{cytoscape}は,遺伝子,タンパク質,化合物などを構成要素とするパスウェイ,ネットワークデータを可視化,統合,解析するためのオープンソースのソフトウェアである.
本システムではCytoScapeのJavascript版であるCytoscape.jsを用いてNeo4jにあるグラフデータをWebアプリケーション上に可視化している.
Cytoscape.jsでは,ノードやエッジに任意の属性を設定できるため,この設定によってコンセプトマップのノード,リンク,リンクキーワードをグラフデータベースへと保存できる.
また,グラフデータをJSON形式で出力可能でそのJSONファイルからNeo4jにデータを受け渡している.
