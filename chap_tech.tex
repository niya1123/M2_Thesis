\section{本章の概要}
本章では,本研究で使用している技術について述べる.

\ref{sec:docker}節では,本システムの構築とPC上で利用するDockerについて述べる.
\ref{sec:python}節では,本システムで使用するサーバサイド言語のPythonとそのライブラリであるFlaskとDjangoについて述べる.
\ref{sec:neo4j}節では,本システムで使用するグラフデータベースであるNeo4jについて述べる.
\ref{sec:cyto}節では,本システムで利用するグラフデータベース可視化ライブラリであるCytoScapeについて述べる.

\section{Docker}\label{sec:docker}
Docker\cite{docker}は,Docker社が開発してるコンテナ仮想化を用いてアプリケーションの開発や実行をするためのオープンプラットフォームである.コンテナ仮想化とは,ホストOSのカーネルを利用してプロセスやユーザを隔離して動かすことで開発環境や実行環境を構築するためのものである.ハイバーバイザを利用した仮想化ソフトウェアとは違い,ハードウェアを仮想化しないため軽量で高速に起動や停止することが可能である.

本システムでは,Dockerを利用してWebアプリケーションを構築している.このため,実運用されているPC環境に影響を及ぼさずに本システムが利用可能となっている.またDockerイメージの共有をすることで特定のOS依存することなく本システムの可能となっている.

\section{Python}\label{sec:python}

\section{Neo4j}\label{sec:neo4j}

\section{CytoScape}\label{sec:cyto}