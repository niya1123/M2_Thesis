\subsubsection{Go}
Go\cite{golang}は静的型付けの効率と安全性,動的型付けのプログラミングの容易さを兼ね備えたプログラミング言語である.
また,Goのコンパイラ,ツール,およびソースコードはすべてフリーなオープンソースである.


\subsubsection{PostgreSQL}
PostgreSQL\cite{postgres}とは,Linux,Windows,MacOS Xに対応した,オープンソースのリレーショナルデータベース管理システム(RDBMS)である.
特徴として以下のようなものがある.
\begin{description}
    \item[拡張性]\mbox{}\\
        PostgreSQLでは,アーキテクチャが拡張可能である.このことにより,データベース内の手順のカスタマイズや自動化に使用できるユーザ定義関数とサードパーティライブラリへのサポートが可能となる.
    \item[同時実行性]\mbox{}\\
        PostgreSQLでは,マルチバージョン同時実行制御があり,ユーザは同じデータベースの書き込みと読み取りが可能となっている.
    \item[標準SQLへの準拠]\mbox{}\\
        PostgreSQLでは,ISO/IECの標準SQLにならった実装となっている.
\end{description}


\subsubsection{Chart.js}
Chart.js\cite{chartjs}とは,HTML5のCanvasを使用したグラフ描画ライブラリである.
Chart.jsはオープンソースなプロジェクトで,8つのチャートタイプと優れたレスポンシブを有している.
8つのチャートタイプは以下の通りである.
\begin{itemize}
    \item 線グラフ
    \item 棒グラフ
    \item レーダーチャート
    \item 円グラフ
    \item 鶏頭図
    \item バブルチャート
    \item 散布図
    \item エリアチャート
\end{itemize}
また,2種類以上のチャートを複合した複合グラフも作成できる.


\subsubsection{Json}
Json\cite{json}(Javascript Object Notation)は,軽量なデータ交換フォーマットである.
人間にとって読み書きが容易で,マシンにとってもパースや生成を行える.

\subsubsection{NGINX}
NGINX\cite{nginx}は,フリーでオープンソースなWebサーバである.
HTTP,HTTPS,SMTP,POP3,IMAPのリバースプロキシ機能や,ロードバランサ,HTTPキャッシュなどの機能を持つ.

\subsubsection{Gunicorn}
Gunicorn\cite{gunicorn}は,UNIXのためのPython WSGI HTTP Serverである.
WSGI\cite{wsgi}とは,PythonにおいてWebサーバとWebアプリケーションを接続するための,標準化されたインターフェース定義のことである.
