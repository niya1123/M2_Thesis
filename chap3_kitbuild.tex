\section{本章の概要}
本章では,キットビルド概念マップについて述べる.
キットビルド概念マップ\cite{kit}\cite{kit2}は,教授者が内容理解構造として作成した概念マップを分解・部品化して学習者に提供し,学習者は提供された部品を組立てることで概念マップを作成する.
組立てられた概念マップは,元の概念マップと重畳することで差分抽出が可能であり,また,複数のマップの重畳することによる集団としてのマップ作成も可能となっている.
これによりキットビルド概念マップは,内容理解構造の全般的で直接的な表出と,評価の自動化を実現する手段として使われている.

以降,キットビルド概念マップの特徴について述べる.

\section{キットビルド概念マップの特徴}
授業理解の過程において,Kiewra\cite{kiewra}やArmbruster\cite{armbruster}は情報の取得と情報館付けによる構造化の二つの過程より成立するとした.
構造化の過程が理解に対してより重要であると分析している.
また,情報の取得に失敗した場合,構造化では補完できない場合が多いため,構造化対象となる情報を学習者に明示的に示し,学習者には構造化に注力させるべきであるとしている.
キットビルド概念マップでは,構造化の対象となる情報を部品として学習者に提供することによって,構造化を保持しつつ情報取得失敗時による構造化失敗を回避できるという特徴がある.

また,教材内容の理解を教授者が概念マップとして明示的に記述することが求められる.
このことから概念マップとして表現できない深い学習内容に対しては表面的な理解に留まるような形でしか表現できない.
しかし,深い学習内容に至る前提としてキットビルド概念マップで表面的な内容理解で使用できるという点ではキットビルド概念マップの有用性は損なわれない.

同様にして,教授者が適切な概念マップを作成できるとは限らず,また,唯一の正解である概念マップを作製できるわけではない.
しかし,キットビルド概念マップでは,教授者が作成した概念マップと学習者が作成した概念マップを比較し,形成的評価・フィードバックを受ける,
すなわち学習者が作成した概念マップにおいて再構成ができていない部分や,同様の誤りが多い部分が存在した場合,教授者側が作成した概念マップに誤りがあると考えられるという点から概念マップの修正を重ねてより正確な概念マップを作成できる.

以上のようにしてキットビルド概念マップには,学習者には学習内容の構造化に注力させ,その構造化自体に誤りがあったとしてもそれを修正していけるような形で作成されているマップであると言える.